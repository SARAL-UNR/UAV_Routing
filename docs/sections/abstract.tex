An unmanned aerial vehicle (UAV) is an autonomous tool capable of accomplishing various time constrained tasks. In order for a UAV to visit all of its targets and completed all of the required tasks in a minimal amount of time, a route must be calculated in advance so the algorithms must be sufficiently fast, even when navigating a large number of targets. UAV routing is a difficult question since it must account for the various constraints as defined by both the UAV and the scenario. This paper solves the problem of UAV routing in which the UAV must periodically visit targets in which each posses a repeating time window the UAV must visit. The UAV has limited fuel and must also recharge at a depot which is it starting position.  To solve the problem, we transform the problem space by adding dummy targets and depots that can only be visited once. To solve the transformation, both a mixed integer linear programming algorithm and a heuristic were created.  